\section{Overview of the VELS system} \label{velsoverview}
The VELS system is an e-learning system for students who are learning VHDL. The
interaction between students and the system is solely via email. Configuration of the
system is done with a configuration file (for configuration items that can not change
during runtime) and a web interface (for configuration items that can change dynamically).

The VELS system can roughly be divided into 3 parts:
\begin{enumerate}
    \item The autosub submission system.
    \item The configuration and status web interface VELS\_WEB.
    \item The individual tasks.
\end{enumerate}

The autosub submission system has the following responsibilities:
\begin{itemize}
    \item Fetch, process and send e-mails.
    \item Store and manage tasks, users and submissions.
    \item Initiate generation of tasks and testing of submissions.
    \item Produce statistics.
\end{itemize}

The configuration and status web interface VELS\_WEB offers the following for a course
operator:
\begin{itemize}
    \item Configure a task queue for the course.
    \item Whitelist, create, edit and view users and their course progress.
    \item Change general configuration parameters of the system.
    \item Modify the messages that get sent to the students at certain events.
    \item View statistics about e-mails and task submission success.
\end{itemize}

The responsibilities of a specific task are the following:
\begin{itemize}
    \item Generate random task parameters.
    \item Generate an appropriate task description and the files that shall
        be provided to the student.
    \item Generate testbenches for task submissions.
    \item Test student submissions and offer meaningful error messages.
\end{itemize}

The following subsections aim to describe the VELS parts internals in order to develop an
understanding what configurations are needed for the VELS system.

\newpage

\subsection{The autosub submission system} \label{sub:autosub}
The autosub submission system is a daemon that is composed of multiple threads
communicating over queues. This multi-threading approach is especially important
for the testing process. As the test of a single task can take rather long and
each student can hand in multiple (correct and incorrect) solutions. Furthermore
the goal is to assure, that there still is progress, even if a test is running.

Therefore the following threads are implemented:

\begin{description}
\item [fetcher: ] The fetcher's purpose is to check the mailbox for incoming 
    e-mails using IMAP. If a new e-mail is in the mailbox the fetcher
    interprets the header and performs some basic checks (e.g. Is the user on 
    the Whitelist?). It then triggers other tasks to act on the e-mail.

    One further purpose of the fetcher is to archive e-mails that have already
    been handled. If a response to an e-mail was sent, the message-id of the
    original e-mail is passed back to the fetcher who then moves the original
    e-mail from the Inbox to the directory used for archiving of e-mails.
\item [sender: ] The sender thread sends out response e-mails based using SMTP. 
    The content of the e-mail depends on the type, the necessary information 
    is provided by the thread that triggers the response.

\item [generator: ] The generator thread generates new tasks for students. This
    way each student will get its own individual task. The generator calls the
    individual generator executables.

\item [activator: ] As tasks can have an arbitrary start time, someone has to check
    whether or not this start time has already come and if the task shall
    be active. This is done by the activator thread -- it periodically checks
    for all tasks that are not active yet, whether or not its start time
    has already come.

\item [workers: ] The worker threads are the ones that check a submitted solution.
    Since testing a solution can take a considerable amount of time, multiple
    workers threads are created. The number of workers is configured via the
    configuration file (see Section \ref{sub:configfile}), ideally this is set
    to something like 2x Number of CPU-Cores. The workers call the
    individual test executables.
\item [dailystats: ] The dailystats thread is used to get a timeline of some very
    basic statistics. Currently the number users, sent e-mails, received e-mails
    and received questions over time are evaluated and plotted into a graph.
    This graph can e.g. be viewed in the Statistics tab of the web interface.
\end{description}

For storing data two databases are used. The name and location of those databases can be configured
through the configuration file, in the following we will refer to them with the default
name that is used, if no other name is provided in the configuration file:
\begin{description}
\item [semester.db: ] The {\tt semester.db} database is used to store all data that is
    specific to one semester. This includes, users, parameterization of tasks assigned
    for users, statistics, etc. The {\tt semester.db} default location is in the root
    autosub directory {\tt "autosub/src"}.
\item [course.db: ] The {\tt course.db} database consists of configuration data that will
    very likely be reused in multiple semesters. This separation makes it very easy
    to start a new semester: just backup {\tt semester.db} to some safe location and
    remove it in the original location (a new empty one will be made automatically upon 
    starting the autosub daemon).Then use the VELS web interface to perform the small 
    modifcations(year/semester,deadlines, etc.) needed in {\tt course.db}. The
    {\tt course.db} default location is in the root autosub directory {\tt "autosub/src"}.
\end{description}

In addition there is one important directory that is used by autosub to store data in,
the directory: {\tt autosub/src/users}. In this directory a unique directory for every
user is created, and this directory the description of each individual received task and submissions 
for the individual tasks are stored. Every directory has the form 
{\tt "autosub/src/users/<user\_id>/Task<nr>/"}. Submissions are stored in separate folders which are named 
{\tt "Submission<nr>\_<date>\_<time>/"}. Task description files are stored in the folder named {\tt "desc/"}.

\subsection{The web inteface VELS\_WEB} \label{sub:VELS_WEB}

The configuration and status web interface VELS\_WEB was designed as an easy interface
for a course operator to view and change course parameters and monitor the course progress.
It was only designed for internal operator use! Therefore it should only reachable via 
the internal institute network using a web browser.

The menu items of the VELS\_WEB are the following:
\begin{description}
\item [Start:] The start page.
\item [Users:] View of the registered students, allows to change or a student via the
    "Edit" button, delete a student via the "Delete" button  and show the progress of
    a student in the course via the "View" button. The table is sortable by column by
    clicking on the head of the column.
\item [Tasks:] The task configuration, will be discussed in Section \ref{sub:configTasks}.
\item [Whitelist:] Change whitelisting for students, will be discussed in Section
    \ref{sub:whitelisting}.
\item [General Config:] Change dynamic configuration items of VELS, will be discussed in
    Section \ref{sub:generalconfig}.
\item [Special Messages:] View and change messages the student gets sent at special events
    (e.g. welcome, not registered, usage, etc.).
\item [Statistics:] View statistics about sent and received emails and task success of
    the students.
\item [User Tasks:] View mappings from students to individual tasks.
\end{description}


\subsection{The tasks structure} \label{sub:TasksStructure}
Every task is located in an individual folder in {\tt "autosub/tasks/implementation"}. Every task is 
composed of the following elements:

\begin{tabular}{|p{3cm}|p{10cm}|}
\hline

Minimal  & \begin{itemize}
    \item {\bf Generator executable:} Generates random parameters for the task, creates entity 
        and behavioral vhdl files and description test and pdf file for the student. Stores the 
        individual task parameters in the \textit{UserTasks} database table.  
    \item {\bf Tester executable:} Given the individual task parameters, tries to compile the 
        student's solution. Generates an individual test bench for the student's solution. Tests 
        the solution and creates feedback text and files for the student. 
    \item {\bf description.txt:} Contains a textual task description. The text in description.txt 
    	is sent to the user as the body of the e-mail that contains a new task description.
	\end{itemize} 
\\
\hline
\end{tabular}

\begin{tabular}{|p{3cm}|p{10cm}|}
\hline

Optional & \begin{itemize}
    \item {\bf Scripts:} Scripts that are called from the executables in order to aid the 
        generation or test process
    \item {\bf Templates:} These files have to be filled with parameters and can be used to generate entity 
        declarations, test benches or description texts and files for the students.
    \item {\bf Static:} These files are static for the task, therefore the same for every student.
    \item {\bf Exam:} Files that are needed for the VELS exam mode.
\end{itemize} 
\\
\hline
\end{tabular} 


A sequence for the task generation can go as follows:
\begin{enumerate}
    \item The generator executable is called by the autosub generator thread.
    \item The generator executable calls a task generation script
        {\tt "scripts/generateTask.py"}.
    \item The task generation script generates random task parameters and returns
        them, fills a LaTeX description template and vhdl templates and stores
        the results in the directory {\tt "tmp/"}.
    \item The generator executable generates the task description pdf file from the
        filled template and copies it, all filled vhdl files and static vhdl files
        to the users task description path.
        {\tt "autosub/src/users/<user\_id>/Task<task\_nr>/desc/"}.
    \item The generator executable stores the path to files that need to be attached
        to the task email for the student in the {\tt semester.db UserTasks}
        database table.
\end{enumerate}

The usual sequence for task submission testing is the following:
\begin{enumerate}
\item The tester executable is called by a autosub worker thread.
\item The tester executable calls a testbench generation script
    {\tt "scripts/generateTestBench.py"}.
\item The tesbench generation script creates a testbench with test vectors and returns it.
\item The tester executable stores the testbench and copies needed files from the task.
    description in the taks directory {\tt "autosub/src/users/<user\_id>/Task<task\_nr>"}
\item The tester executable checks if the appropiate submission files are present.
    These files are copied to the user- task- folder by the fetcher thread.
\item The tester executable analyzes files generated by the task generation, if errors
    occur the process stops and sends an email to student and system administrator.
\item The tester executable analyzes the student submission files, if errors occur they
    are written to a {\tt error\_msg} file and the process stops.
\item The tester elaborates and runs the testbench, if the tests fail meaningful messages
    are written to a {\tt error\_msg} file and the process stops.
\item The tester returns success.
\end{enumerate}
