\section{Truth Table} \label{truthtable}
    \begin{tabular}{|p{2cm}|p{11cm}|}
        \hline
        Overview & The student has to program the behavior of an entity which behaves according to a given 
        truth table. The student learns how computations on inputs can be done and and the proper output 
        can be set.
        \\
        \hline
        Description & The student receives pdf file with a truth table for 4 inputs and 1 output. 
        It is noted that the task can be solved via look up table or prior simplification via KV
		(Karnaugh Veith)
        algorithm. The student receives a vhdl file with the entity declaration, which should 
        not be changed and a vhdl file with for the student's behavioral implementation. 
        \\
        \hline
        Creation & In a 4 input system the truth table is composed of $N=2^4$ rows. Therefore 
        $2^N=2^{16}=65536$ possible truth tables can be created. In order to allow  KV optimizations 
        as a DNF(Disjunctive Normal Form) the truth table is generated and then checked if optimization 
		is possible using the Quine–McCluskey algorithm.
        \\
        \hline
        Submission & The student has to submit his behavioral implementation vhdl file.
        \\
        \hline
        Testing & A testbench tests the students entity with all 16 possible input combinations and compares 
        them to the given truth table via a lookup table. 
        \\
        \hline
        Feedback & If the syntax analysis returned errors, they are sent to the student. If one of the 
        tests results in a wrong result, the chosen operands and the proper result is sent to the student. 
        The student is informed if his solution was proper or not. 
        \\
        \hline 
    \end{tabular}

\newpage

\section{Gates} \label{basicgates}
    \begin{tabular}{|p{2cm}|p{11cm}|}
        \hline
        Overview & The student has to program the behavior of an entity which behaves according to a given
        gate network. The student learns how to use predefined components and connect them. 
        \\
        \hline
        Description & The student receives a pdf file with a gate network as a picture. It is noted that the 
        tasks should be solved using IEEE 1164 gates. The student receives a vhdl file 
        with the entity declaration, which should not be changed and a vhdl file with for the behavioral
        implementation. The to be used IEEE 1164 gates are supplied in the form of the following files: a 
        file for the entities, a file for the behavior and a package file.
        \\
        \hline
         Creation & The randomly generated gate network has the following properties:
         \begin{itemize}
             \item 2 level depth
             \item maximum of 4 gates per level
             \item 4 inputs, 1 output
             \item gate types: AND, NAND, OR, NOR, XOR, XNOR
             \item every input can be negated 
        \end{itemize}
        The generated network is converted to a pdf file using \LaTeX. 
        \\
        \hline
        Submission & The student has to submit his behavioral implementation vhdl file.
        \\
        \hline
        Testing & A testbench tests the students entity with all 16 possible input combinations 
        and compares the output them to server side calculated proper values.
        \\
        \hline
        Feedback & If the syntax analysis returned errors, they are sent to the student. If one of the 
        tests results in a wrong result, the chosen operands and the proper result is sent to the student. 
        The student is informed if his solution was proper or not. 
        \\
        \hline 
    \end{tabular}

\newpage

\section{PWM(Pulse-Width Modulation) Generator}\label{pwmgenerator}
    \begin{tabular}{|p{2cm}|p{11cm}|}
        \hline
        Overview &  The student has to program the behavior of an entity which produces a PWM 
        signal with a given output frequency and a given duty cycle. The output PWM signal shall 
        be generated with the help of a given input clock signal at 50 MHz. The student 
        learns how to handle clock signals for output signal generation.
        \\
        \hline
        Description & The student receives information about what PWM is, the frequency of the input 
        clock (50 MHz), the frequency of the desired PWM output signal and the desired duty cycle. 
        The student receives a vhdl file with the entity declaration, which should not be changed 
        and a vhdl file with for the behavioral implementation. The student is informed that the
                use of the keywords 'after' and 'wait' is prohibited for this task.
        \\
        \hline
        Creation & Both the output frequency and duty cycle are randomly generated. The input clock is fixed
        at 50 MHz.
        \\
        \hline
        Submission & The student has to submit his behavioral implementation vhdl file.
        \\
        \hline
        Testing & A test bench feeds the student's entity with the specified input frequency, tries to 
        calculate a frequency and a duty cycle of the output signal and compares it to the specified 
        frequency and duty cycle. Students' entities which use the keywords 'after' and 'wait' are rejected. 
        \\
        \hline
        Feedback & If the syntax analysis returned errors, they are sent to the student. The students' 
        PWM signal is tested for the specified frequency and duty cycle. If the solution is not proper,
        the student is informed of the measured values and receives a gtkwave file containing his output 
        signal. If no continuous signal is detectable, the student is told so and also receives the 
        gtkwave file. The student is informed if his solution was proper or not. 
        \\
        \hline 
    \end{tabular}

\newpage

\section{Arithmetic}\label{arithmetic}
    \begin{tabular}{|p{2cm}|p{11cm}|}
        \hline
        Overview & The student has to program the behavior of an entity which acts as an adder or subtracter
        for 2 operands of a given bit width and a given number representation. The student learns how simple 
        add and subtract operations can be implemented in hardware via vhdl   
        \\
        \hline
        Description & The student receives which type of operation for which operand bit length and number 
        representation has to be implemented. The appropriate flags (carry, overflow) for the given 
        operation are described. The student receives a vhdl file with the entity declaration,
        which should not be changed and a vhdl file the behavioral implementation. 
        \\
        \hline
        Creation & The operation, the bit width of the 2 operands and number representation is randomly 
        chosen from the following choices:
        \begin{itemize}
            \item addition or subtraction
            \item ones' complement, two's complement or unsigned
            \item both operands have the same length between 4 and 16 bit
        \end{itemize}      
        The appropriate flags are included as outputs of the predefined entity.
        \\
        \hline
        Submission & The student has to submit his behavioral implementation vhdl file.   
        \\
        \hline
        Testing & A test bench tests the student's entity with random operands. All error 
        cases (overflow, carry) are at least tested once.
        \\
        \hline
        Feedback & If the syntax analysis returned errors, they are sent to the student. If one of the 
        tests results in a wrong result or the appropriate flags were not set, the chosen operands 
        and the proper result including flags is sent to the student. The student is informed if 
        his solution was proper or not.
        \\
        \hline 
    \end{tabular}

\newpage

\section{FSM(Finite State Machine)}\label{finitestatemachine}
    \begin{tabular}{|p{2cm}|p{11cm}|}
        \hline
        Overview & The student has to program the behavior of an entity which acts as a Mealy Finite 
        State Machine according to a given state diagram. The student learns how a simple Mealy Finite
        State Machine can be implemented with vhdl using synchronous design principles.
        \\
        \hline
        Description & The student receives information about Finite State Machines and synchronous
        design and a pdf file with a picture of a state diagram. It is noted that the whole design should be 
        programmed according to synchronous design. The student receives a vhdl file with the 
        entity declaration, which should not be changed and a vhdl file with the states predefined 
        for the behavioral implementation.
        \\
        \hline
        Creation &  The randomly generated state diagram has the following properties:
        \begin{itemize}
            \item 5 states
            \item 2 input bits, 2 output bits
        \end{itemize}
        For testing the state the state machine is in is also a entity output. The generated state diagram 
        is converted to a pdf file via \LaTeX. 
        \\
        \hline
        Submission & The student has to submit his behavioral implementation vhdl file.
        \\
        \hline
        Testing & A test bench feeds the student's entity with inputs and checks both output and state
        transitions. The state machine is tested extensively to achieve 100\% test coverage. 
        \\
        \hline
        Feedback & If the syntax analysis returned errors, they are sent to the student. If the end 
        state was not reached the student receives information about the last valid state and the next 
        expected state. The student is informed if his solution was proper or not.
        \\
        \hline 
    \end{tabular}

\newpage

\section{CRC(Cyclic Redundancy Check Generator)}\label{crcgenerator}
    \begin{tabular}{|p{2cm}|p{11cm}|}
        \hline
        Overview & The student has to program the behavior of an entity which generates the CRC for data 
        of a given bit width with a given generator polynomial. The student learns how shift register can be 
        used for calculations.
        \\
        \hline
        Description & The student receives an explanation about CRC a bit width, a generator 
        polynomial and an example input and result. It is noted that shift registers should
        be used for implementation. The student receives a vhdl file with the entity declaration,
        which should not be changed and a vhdl file the behavioral implementation. 
        \\
        \hline
        Creation & The generator polynom and the message bit length is randomly generated. The message 
        bit length is chosen between 12 and 24 bit, the generator polynomial degree is chosen between 5 
        and 11. It is assured that the degree of the generator polynomial is smaller than the message 
        bit length. 
        \\
        \hline
        Submission & The student has to submit his behavioral implementation vhdl file. 
        \\
        \hline
        Testing & A test bench feeds the student's entity sequentially with multiple data words and checks the 
        resulting CRC values.
        \\
        \hline
        Feedback & If the syntax analysis returned errors, they are sent to the student. If one of the 
        tests results in a wrong CRC value, the chosen data, the wrong CRC and the proper
        solution are sent to the student. The student is informed if his solution was proper or not.
        \\
        \hline 
    \end{tabular}
 
\newpage

\section{ALU} \label{alu}
    \begin{tabular}{|p{2cm}|p{11cm}|}
        \hline
        Overview & The student has to implement the behavior of ALU based on the instructions which are randomly chosen. 
        The student learns how to prepare VHDL pre-defined components and also learns how to use inputs and outputs.
        \\
        \hline
        Description & The student receives a PDF file which describes the instructions and the chosen flag for each one. 
        She has to write the code in a VHDL file with the entity declaration, and she should not change it. 
        \\
        \hline
        Creation & Tasks are parametrized by the following parameters:
		\begin{itemize}
			\item 4 instructions out of \{ADD, SUB, AND, OR, XOR, Shift Left (SHL), Shift Right (SHR), Rotate Left (ROL), Rotate Right (ROR), comparator\}
			\begin{itemize}
				\item One instruction among ADD and SUB
				\item Two instructions among AND, OR and XOR and comparator
				\item One instruction among SHL, SHR, ROL and ROR 
			\end{itemize}   
			\item One flag out of \{overflow, zero, sign, carry, odd parity\}
			\begin{itemize}
				\item one flag out of \{overflow, zero, sign, carry\} for ADD or SUB
				\item one flag out of \{zero, sign, odd parity\} for AND, OR or XOR
				\item the flag is updated for SHL, SHR, ROL, ROR and comparator during the operation
			\end{itemize} 
			\item Rising or falling edge for the clock signal
		\end{itemize}  
	The length of data is 4 bits for all students.
        \\
        \hline
        Submission & The student has to submit her behavioral implementation VHDL file.
        \\
        \hline
        Testing & A test-bench is generated based on generated parameters. The correct output is generated by Python and is 
        added to the VHDL test bench. Each instruction is checked, and the result from the student's code is compared with 
        the correct result. Then, clock and enable signals are checked to work properly.
        \\
        \hline
    \end{tabular}
    \begin{tabular}{|p{2cm}|p{11cm}|}
        \hline
        Feedback & Student's code is analyzed to see if there is any syntax error. The error is sent to the student by email. 
        It is also simulated to be checked if it works properly. For each instruction, the first wrong error is reported: the wrong 
        and expected output and correspondent input is sent to the student. It is said the problem may be due to the incorret 
        implementation or incorrectly using the signals and variables. Then, the student is notified if there is an error in 
        using the clock signal. Finally, she receives an email if there is an error in using the enable signal.
        \\
        \hline 
    \end{tabular}

\newpage

\section{ROM} \label{rom}
    \begin{tabular}{|p{2cm}|p{11cm}|}
        \hline
        Overview & The student learns how to define arrays and allocate a value to each component of that array. 
        He also learns to define the length of inputs.
        \\
        \hline
        Description & The student receives a PDF file with a specification of the ROM. He also receives a VHDL code in which he should 
        implement the behavior of ROM. He also fills the ROM with the values which are randomly generated. He should be careful not to 
        change the input and output deceleration and not to add unnecessary letters or space.

		ROM needs to be filled with the specified data. A specific amount of ROM size is filled with randomly generated data in a binary representation. 
		The student should write them in his code as the data existing in the ROM. The rest of location is filled with zero in a binary format.
        \\
        \hline
        Creation & ROM has the following randomly generated parameters:
		\begin{itemize}
			\item Clock cycle: falling or rising edge
			\item Length of address (between 8 to 16 bits)
			\item Length of instruction (between 8 to 16 bits)
			\item Number of data existing in the ROM (between 12 to 20 bits) 
			\item Start location of existing instructions (between 50 to 200)
			\item Instructions which exist in the ROM are different from one student to another
		\end{itemize}  
        \\
        \hline
        Submission & The student has to submit his behavioral implementation VHDL file.
        \\
        \hline
        Testing & A test bench is generated based on the parameters which have been chosen for the student. First, the content of each address 
        is checked in the test-bench randomly to see whether the student correctly filled the ROM with the specified data or not. Then, the clock 
        cycle and enable signals are checked to work correctly.
        \\
        \hline
        Feedback & If the syntax analysis returns errors, they are sent to the student. If the VHDL simulation does not work properly, the wrong 
        and expected outputs and correspondent inputs are sent to the student. If the clock signal is defined wrong, it is notified to him.
        \\
        \hline 
    \end{tabular}

\newpage

\section{RAM} \label{ram}
    \begin{tabular}{|p{2cm}|p{11cm}|}
        \hline
        Overview & The student has to program a RAM with a specified size, length of instruction and address and also the number of reading and 
        writing operations at the same time. 
        \\
        \hline
        Description & The RAM has two address lines, one or two input data, and one or two output data lines and also read/write enables. 
        The ports of the RAM are provided in VHDL code. Based on the number of reading and writing actions, there are four states, one of 
        which is randomly selected for the student. The chosen state has a few behaviors. The student should implement the RAM based on these behaviors. 
        \\
        \hline
        Creation & Parameters:
		\begin{itemize}
			\item Length of each memory cell (an even number between 8 to 20)
			\item Length of address (6 to 12-bit length)
			\item Number of reading and writing actions:
			\begin{itemize}
				\item one reading and two writing operations
				\item two reading and one writing operations
				\item two reading and two writing operations
				\item one reading and one writing operations at the same time (even to the same address)
			\end{itemize}   
			\item Length of input: word or double word
			\item Length of output: word or double word
		\end{itemize}  
		The result should be produced on the rising edge of the clock signal for all students.
        \\
        \hline
        Submission & The student has to submit his behavioral implementation VHDL file.
        \\
        \hline
        Testing & Four test benches are provided based on those four states. In other words, each test bench is specific to the number of reading 
        and writing operations. 16 cells of the memory are randomly selected during the test bench generation, and two sets of 16 data are also 
        randomly generated. Then, the related test bench is generated. The test bench first checks the writing process for each address in RAM and 
        then reading process for each address. The tech bech inclues several parts to test each behavior specified in the description file.
        \\
        \hline
    \end{tabular}
    \begin{tabular}{|p{2cm}|p{11cm}|}
        \hline
        Feedback & If the syntax analysis returns errors, they are sent to the student. If the written value cannot be fetched in a reading operation, 
        an error message is sent to the student which tells him that there is an error in writing or reading operation to a specific location. If two 
        reading or two writing processes to different addresses is specified for the student, the result of testing it is sent to the student as well. 
        The result of testing other specifications like ‘double-word’ also is sent. 
        \\
        \hline 
    \end{tabular}

\newpage
