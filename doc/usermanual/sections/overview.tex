\section{Manual Overview} \label{overview}

The following document aims to inform an operator on on how the autosub system works
and how to configure and run a course using the autosub system. This manual is written 
primarily for using autosub with VELS (VHDL E-Learning System). Nevertheless the autosub 
system and the described structures are designed to be usable for diverse E-Learning 
purposes.

The first part of this manual (Section \ref{system_interaction}) outlines the 
interaction and use cases for a user and an operator with the E-Learning system.

The second part of this manual describes VELS, which itself can roughly be divided 
into 3 parts:
\begin{enumerate}
    \item The autosub submission system (see Section \ref{autosub_system})
    \item The configuration and status web interface VELS\_WEB (see Section 
	      \ref{VELS_WEB})
    \item The tasks interface (see Section \ref{tasks_system})
\end{enumerate}

The third part of this manual depicts prequesites (see Section \ref{system_prerequisites})
and whicht steps have to be taken for system setup (see Section \ref{system_setup}).

The fourth part describes how to create new tasks for VELS (see Section \ref{create_new_tasks}).

The Appendix (Section \ref{appendix}) list Implementational details of VELS.
