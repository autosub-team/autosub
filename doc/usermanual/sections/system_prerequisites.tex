\section{VELS Pre-Requisites} \label{system_prerequisites}

In the following, the installation of pre-requisites for autosub is documented.

\subsection{Debian 7 (Wheezy)}

The probably easiest step is to install standard debian packages:

\begin{verbatim}
sudo apt-get install python3 python3-mock python3-pip python3-nose 
sudo apt-get install zip flip git gnat-4.6-base libgnat-4.6 
sudo apt-get install texlive-latex-extra pgf graphviz build-dep
sudo apt-get install zlib1g-dev
\end{verbatim}

Unfortunately there is a small bug in one of the latex packages that we use,
therefore you need to apply a patch to that package:

\begin{verbatim}
cd /usr/share/texlive/texmf-dist/tex/generic/pst-circ
patch < /path/to/autosub/doc/patch/pst-circ.tex.patch
\end{verbatim}

In case gates in the pdfs you are not beautiful as you are used to, it might have
happended that an update was applied over that patch and you need to re-apply it (sorry for that little mess).


Next we need to upgrade easy\_install for python3 to a newer version -- this
is the pre-requisite for the installation of the matplotlib for python3.

\begin{verbatim}
easy_install3 -U distribute
\end{verbatim}

And finally we may now install some libraries that are not (yet) available in
the debian package system via using pip:

\begin{verbatim}
pip-3.2 install pyqt5
pip-3.2 install matplotlib
pip-3.2 install bitstring
pip-3.2 install graphviz
\end{verbatim}

Last, we install a vhdl simulator. If you want to use ghdl you may download
the debian package from sourceforge \footnote{http://sourceforge.net/projects/ghdl-updates/files/Builds/ghdl-0.31/Debian/} and install it using dpkg:

\begin{verbatim}
dpkg -i Downloads/ghdl_0.31-2wheezy1_amd64.deb
\end{verbatim}

All the dependencies of that package have already been resolved by the above
list of packages installed with apt.

To use the Simulator of Xilinx's ISE: ISim, see Section \ref{ISE-install}

At this point you may run the test-suite, as explained in Section \ref{sub:testingvels}.
If those tests run through without complications, everything you need to run the VELS
E-Learning system has been installed successfully.

\subsection{Debian 8 (Jessie)}

For jessie everything is even easier as more libraries are available for python3 in the
apt package pool now:

\begin{verbatim}
sudo apt-get install python3  python3-mock python3-pip python3-nose
sudo apt-get install zip flip git gnat-4.9-base libgnat-4.9
sudo apt-get install texlive-latex-extra pgf graphviz  build-dep
sudo apt-get install zlib1g-dev python3-pyqt5 python3-matplotlib  
\end{verbatim}

Unfortunately there is a small bug in one of the latex packages that we use,
therefore you need to apply a patch to that package:

\begin{verbatim}
cd /usr/share/texlive/texmf-dist/tex/generic/pst-circ
patch < /path/to/autosub/doc/patch/pst-circ.tex.patch
\end{verbatim}

In case gates in the pdfs you are not beautiful as you are used to, it might have
happended that an update was applied over that patch and you need to re-apply it (sorry for that little mess).

So now, only two more packages that need to be installed using pip:

\begin{verbatim}
pip3 install bitstring
pip3 install graphviz
\end{verbatim}

Last, we install a vhdl simulator. If you want to use ghdl you may download
the debian package from sourceforge \footnote{http://sourceforge.net/projects/ghdl-updates/files/Builds/ghdl-0.33/debian/}
and install it using dpkg.

\begin{verbatim}
dpkg -i ghdl_0.33-1jessie1_amd64.deb
\end{verbatim}

To use the Simulator of Xilinx's ISE: ISim, see Section \ref{ISE-install}

At this point you may run the test-suite, as explained in Section \ref{sub:testingvels}.
If those tests run through without complications, everything you need to run the VELS
E-Learning system has been installed successfully.

\subsection{Debian 9 (Stretch)}

For stretch the standard debian packages are available for installation by:

\begin{verbatim}
sudo apt-get install python3 python3-mock python3-pip python3-nose
sudo apt-get install zip flip git gnat-6 libgnat-6
sudo apt-get install texlive-latex-extra pgf graphviz build-dep
sudo apt-get install zlib1g-dev python3-pyqt5 python3-matplotlib 
\end{verbatim}

Unfortunately there is a small bug in one of the latex packages that we use,
therefore you need to apply a patch to that package:

\begin{verbatim}
cd /usr/share/texlive/texmf-dist/tex/generic/pst-circ
patch < /path/to/autosub/doc/patch/pst-circ.tex.patch
\end{verbatim}

In case gates in the pdfs you are not beautiful as you are used to, it might have
happended that an update was applied over that patch and you need to re-apply it (sorry for that little mess).

So now, only two more packages that need to be installed using pip:

\begin{verbatim}
pip3 install bitstring
pip3 install graphviz
\end{verbatim}

Last, we install a vhdl simulator. The simulator of Xilinx's ISE installation is recommended since there is no dedicated
version of vhdl for Debian stretch. 
If you still want to use ghdl you can try the version described in Debian 8 (Jessie).

\begin{verbatim}
dpkg -i ghdl_0.33-1jessie1_amd64.deb
\end{verbatim}

To use the Simulator of Xilinx's ISE: ISim, see Section \ref{ISE-install}

At this point you may run the test-suite, as explained in Section \ref{sub:testingvels}.
If those tests run through without complications, everything you need to run the VELS
E-Learning system has been installed successfully.

\subsection{Instalation Notes for Xilinx ISE 14.7}\label{ISE-install}

An additional option, to using the open-source software ghdl as the VHDL Simulation tool for VELS, is to use the Simulator of Xilinx's ISE: ISim. The ISE (``Full Installer for Linux'') can be downloaded from Xilinx's official download page. It requires registration and licensing agreement, but there is no charge.

For installing ISE to the default location /opt/Xilinx/ where the VELS system expects it to be, you need permission to write to this location. So we will use the user root to call the graphical installer. To do so we need to allow root to use the users X server. As a user run:

\begin{verbatim}
$ xhost +
\end{verbatim}

to allow root (or anyone) to connect to your users X server temporarily. Now as root start the graphical installer, located at the top of the extracted download, with

\begin{verbatim}
# ./xsetup
\end{verbatim}

In case that you are using the KDE desktop environment, you have to remove the \verb!QT_PLUGIN_PATH! environment variable before starting the graphical installer:

\begin{verbatim}
# unset QT_PLUGIN_PATH
\end{verbatim}

If you are running the VELS system on a machine without a graphical user interface, the best way we found to get Xilinx's ISE onto a remote debian server was to copy the entire installation directory after it has been installed on a non-headless machine.
