\documentclass[a4paper,12pt]{article}
\usepackage{a4wide}
\usepackage{tikz}
\usetikzlibrary{calc}
\usepackage{hyperref}

\usepackage[ngerman]{babel}

\begin{document}
\pagestyle{empty}
\setlength{\parindent}{0em}
\section*{CRC (Zyklische Redundanzpr\"ufung)}

Ihre Aufgabe ist es, das Verhalten einer Entity  namens "`crc"' zu programmieren. Die Entity ist in der angeh\"angten Datei "`crc.vhdl"' deklariert und hat folgende Eigenschaften:
\begin{itemize}
\item Eingang: NEW\_MSG vom Typ std\_logic
\item Eingang: MSG vom Typ std\_logic\_vector mit der L\"ange {{MSGLEN}}
\item Eingang: CLK vom Typ std\_logic
\item Ausgang: CRC vom Typ std\_logic\_vector mit der L\"ange {{GENDEG}}
\item Ausgang: CRC\_VALID vom Typ std\_logic
\end{itemize}

\vspace{0.5cm}

\begin{center}
\begin{tikzpicture}
\draw node [draw,rectangle, minimum height=20mm, minimum width=35mm,rounded corners=2mm,thick](entity){};

\draw[->,thick] ($ (entity.west)+(-10mm,7.5mm) $) -- ($ (entity.west) +(0mm,7.5mm)$);
\draw node at ($ (entity.west)+(-23mm,7.5mm) $){NEW\_MSG};

\draw[->,thick] ($ (entity.west)+(-10mm,0mm) $) -- ($ (entity.west) +(0mm,0mm)$);
\draw node at ($ (entity.west)+(-18mm,0mm) $){MSG};

\draw[->,thick] ($ (entity.west)+(-10mm,-7.5mm) $) -- ($ (entity.west) + (0mm,-7.5mm)$);
\draw node at ($ (entity.west)+(-18mm,-7.5mm) $){CLK};

\draw[->,thick] ($(entity.east)+(0mm,+5mm) $) -- ($(entity.east)+(+10mm,+5mm)$);
\draw node at ($ (entity.east) + (17mm,+5mm) $){CRC};

\draw[->,thick] ($(entity.east)+(0mm,-5mm) $) -- ($(entity.east)+(+10mm,-5mm)$);
\draw node at ($ (entity.east) + (24mm,-5mm)$){CRC\_VALID};

\draw node at ($ (entity) - (0,12mm)$){crc};

\end{tikzpicture}
\end{center}

Ver\"andern sie die Datei "`crc.vhdl"' nicht!
\\

Die "`crc"' Entity soll einen CRC-Wert f\"ur eine Nachricht unter der Ber\"ucksichtigung folgender Bedingungen generieren:
\begin{itemize}
\item Nachrichtenl\"ange: {{MSGLEN}}
\item Ordnung des Generators: {{GENDEG}}
\item Generator Polynom: ${{GENSTRING}}$ ({{GENBIN}})
\end{itemize}
Die Berechnung erfolgt dabei nach folgendem Schema:
\begin{enumerate}
    \item Eine steigende Flanke am Port NEW\_MSG zeigt an, dass eine neue CRC Berechnung f\"ur die Nachricht auf Port MSG ausgef\"uhrt werden soll.
    \item Nach {{CLOCKCYCLES}} Taktzyklen des Taktes (Port CLK) soll der berechnete CRC-Wert am Port CRC ausgegeben werden. Au"serdem soll der Port CRC\_VALID auf '1' gesetzt werden, um anzuzeigen, dass ein g\"ultiges Resultat vorhanden ist.
\end{enumerate}

\newpage

Zur Berechnung des CRC-Wertes soll die Entity "`crc"' ein r\"uckgekoppeltes Schieberegister "`fsr"' ("`feedback shift register"') verwenden. Dessen Entity ist in der angeh\"angten Datei "`fsr.vhdl"' deklariert und hat folgende Eigenschaften:
\begin{itemize}
    \item Eingang: EN vom Typ std\_logic
    \item Eingang: RST vom Typ std\_logic
    \item Eingang: CLK vom Typ std\_logic
    \item Eingang: DATA\_IN vom Typ std\_logic
    \item Ausgang: DATA vom Typ std\_logic\_vector mit der L\"ange {{CRCWIDTH}}
\end{itemize}

\vspace{0.5cm}

\begin{center}
\begin{tikzpicture}
\draw node [draw,rectangle, minimum height=20mm, minimum width=35mm,rounded corners=2mm,thick](entity){};

\draw[->,thick] ($ (entity.west)+(-10mm,7.5mm) $) -- ($ (entity.west) +(0mm,7.5mm)$);
\draw node at ($ (entity.west)+(-17mm,7.5mm) $){EN};

\draw[->,thick] ($ (entity.west)+(-10mm,2.5mm) $) -- ($ (entity.west) +(0mm,2.5mm)$);
\draw node at ($ (entity.west)+(-18mm,2.5mm) $){RST};

\draw[->,thick] ($ (entity.west)+(-10mm,-2.5mm) $) -- ($ (entity.west) +(0mm,-2.5mm)$);
\draw node at ($ (entity.west)+(-18mm,-2.5mm) $){CLK};

\draw[->,thick] ($ (entity.west)+(-10mm,-7.5mm) $) -- ($ (entity.west) + (0mm,-7.5mm)$);
\draw node at ($ (entity.west)+(-23mm,-7.5mm) $){DATA\_IN};

\draw[->,thick] ($(entity.east)+(0mm,+0mm) $) -- ($(entity.east)+(+10mm,+0mm)$);
\draw node at ($ (entity.east) + (20mm,+0mm) $){DATA};

\draw node at ($ (entity) - (0,12mm)$){fsr};

\end{tikzpicture}
\end{center}

Ver\"andern sie die Datei "`fsr.vhdl"' nicht!
\\

Das r\"uckgekoppelte Schieberegister soll folgendes Verhalten aufweisen:
\begin{itemize}
    \item Der Port EN aktiviert bzw. deaktiviert die Entity "`fsr"' (Er ist "`active high"').
    \item Eine steigende Flanke am Port RST setzt den gesamten Inhalt des Registers auf 0 zur\"uck.
    \item Bei jeder steigenden Taktflanke von CLK soll die R\"uckf\"uhrungs- (feedback) und Schiebeoperation (shift) ausgef\"uhrt werden, wobei das Bit am Port DATA\_IN hereingeschoben wird.
    \item  Nach jeder R\"uckf\"uhrungs- und Schiebeoperation sollen die neuen Inhalte des Registers am Port DATA anliegen.
\end{itemize}


Programmieren Sie das Verhalten der Entity "`crc"' in der angeh\"angten Datei "`crc\_beh.vhdl"' und das Verhalten der Entity "`fsr"' in der angeh\"angten Datei "`fsr\_beh.vhdl"'.\\
\\
\rule{16cm}{0.4pt}
\\
\textbf{Zusammenfassung: CRC Generierung}
\\
Eine gute Zusammenfassung zum Thema CRC und der Erzeugung des Codes mittels Schieberegister k\"onnen Sie hier finden:\\
\url{http://www.hackersdelight.org/crc.pdf}
\\
\rule{16cm}{0.4pt}
\newpage


Ein Beispiel f\"ur Ihren CRC Generator ${{GENSTRING}}$ ({{GENBIN}}) ist:
\begin{itemize}
    \item MSG: {{EXAMPLEMSG}}
    \item CRC: {{EXAMPLECRC}}
\end{itemize}

\vspace{0.3cm}
Um Ihre L\"osung abzugeben, senden Sie ein E-Mail mit dem Betreff "`Result Task {{ TASKNR }}"' und Ihren Dateien "`crc\_beh.vhdl"' und "`fsr\_beh.vhdl"'  an {{ SUBMISSIONEMAIL }}.
\vspace{0.7cm}

Viel Erfolg und m\"oge die Macht mit Ihnen sein.


\end{document}
