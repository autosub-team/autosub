\documentclass[a4paper,12pt]{article}
\usepackage{a4wide}
\usepackage{tikz}
\usetikzlibrary{calc}
\usepackage{hyperref}
\usepackage[ngerman]{babel}

\begin{document}
\pagestyle{empty}
\setlength{\parindent}{0em}
\section*{\noindent Counter}
Ihre Aufgabe ist es, das Verhalten einer Entity  namens "`counter"' zu programmieren. Die Entity ist in der angeh\"angten Datei "`counter.vhdl"' deklariert und hat folgende Eigenschaften:

\begin{itemize}
	\item Eingang: CLK vom Typ std\_logic
	{{enable_property_desc}}
	{{sync_property_desc}}
	{{async_property_desc}}
	{{input_property_desc}}
	{{overflow_property_desc}}
	\item Ausgang: Output vom Typ std\_logic\_vector und der L\"ange {{counter_width}}

\begin{center}
\begin{tikzpicture}
\draw node [draw,rectangle, minimum height={{minimum_height}} mm, minimum width=35mm,rounded corners=2mm,thick](entity){};
{{inputs_tikz}}
{{outputs_tikz}}
\draw node at ($ (entity) - (0,0mm)$){ counter };
\end{tikzpicture}
\end{center}

\end{itemize}

%\begin{center}

Ver\"andern Sie die Datei "`counter.vhdl"' nicht!\\

Die "`counter"' Entity soll den Ausgangsvektor Output bei {{every_a}} steigenden Flanke des Taktsignals CLK inkrementieren. Der Anfangswert des Ausgangs (vor der ersten steigenden Flanke des Takts) soll "`{{init_value_padded}}"' betragen. Wenn das Sync{{sync_variation}} Signal auf '1' gesetzt ist, dann soll der Ausgang bei der steigenden Flanke des Taktes auf {{sync_text}} gesetzt werden. Wenn das Async{{async_variation}} Signal auf '1' gesetzt ist, dann soll der Ausgang sofort auf {{async_text}} gesetzt werden. {{Enable_Overflow_text}} \\

Programmieren Sie dieses Verhalten in der angeh\"angten Datei "`counter\_beh.vhdl"'.\\

Um Ihre L\"osung abzugeben, senden Sie ein E-Mail mit dem Betreff "`Result Task {{TASKNR}}"' und Ihrer Datei "`counter\_beh.vhdl"'  an {{SUBMISSIONEMAIL}}.

\vspace{0.7cm}

Viel Erfolg und m\"oge die Macht mit Ihnen sein.

\end{document}
