\documentclass[a4paper,12pt]{article}
\usepackage{a4wide}
\usepackage{tikz}
\usetikzlibrary{calc}
\usepackage{hyperref}
\usepackage[ngerman]{babel}

\begin{document}
\pagestyle{empty}
\setlength{\parindent}{0em}
\section*{\noindent Counter}
Ihre Aufgabe ist es, das Verhalten einer Entity  namens "`counter"' zu programmieren. Die Entity ist in der angeh\"angten Datei "`counter.vhdl"' deklariert und hat folgende Eigenschaften:

\begin{itemize}
	\item Eingang: CLK vom Typ std\_logic
	\item Eingang: RST vom Typ std\_logic
	{{enable_property_desc}}
	{{sync_property_desc}}
	{{async_property_desc}}
	{{input_property_desc}}
	{{overflow_property_desc}}
	\item Ausgang: Output vom Typ std\_logic\_vector und der L\"ange {{counter_width}}
\end{itemize}

\begin{center}
\begin{tikzpicture}
{# draw the entity block dynamically #}


\draw node [draw,rectangle, minimum height={{minimum_height}}mm, minimum width=35mm,rounded corners=2mm,thick](entity){};

    
    \draw[->] ($ (entity.west) + (-10mm, {{ (minimum_height / 2) - current_tikz_offset}} mm)$) -- ($ (entity.west) + (0mm,{{(minimum_height / 2) - current_tikz_offset}}mm)$);
    \draw[anchor=east] node at ($ (entity.west) + (-9mm,{{(minimum_height/2) - current_tikz_offset}}mm)$){ {{input_names[i-1]}} };



    
    \draw[->] ($ (entity.east) + (0mm,{{(minimum_height / 2) - current_tikz_offset}}mm)$) -- ($ (entity.east) + (10mm,{{(minimum_height / 2) - current_tikz_offset}}mm)$);
    \draw[anchor=west] node at ($ (entity.east) + (9mm,{{(minimum_height/2) - current_tikz_offset}}mm)$){ {{output_names[i-1]}} };


\draw node at ($ (entity) - (0,0mm)$){ counter };
\end{tikzpicture}
\end{center}


Ver\"andern Sie die Datei "`counter.vhdl"' nicht!\\

Die "`counter"' Entity soll den Ausgangsvektor Output bei {{every_a}} steigenden Flanke des Taktsignals CLK inkrementieren. Der Eingang RST ist ein synchroner Reset, der Initialwert am Ausgang Output nach dem Reset soll "`{{init_value_padded}}"' betragen. Wenn der Eingang Sync{{sync_variation}}auf '1' gesetzt ist, dann soll der Ausgang Output bei der steigenden Flanke des Taktes auf {{sync_text}} gesetzt werden. Wenn der Eingang Async{{async_variation}} auf '1' gesetzt ist, dann soll der Ausgang Output sofort auf {{async_text}} gesetzt werden. {{Enable_Overflow_text}} \\

Programmieren Sie dieses Verhalten in der angeh\"angten Datei "`counter\_beh.vhdl"'.\\

Um Ihre L\"osung abzugeben, senden Sie ein E-Mail mit dem Betreff "`Result Task {{TASKNR}}"' und Ihrer Datei "`counter\_beh.vhdl"'  an {{SUBMISSIONEMAIL}}.

\vspace{0.7cm}

Viel Erfolg und m\"oge die Macht mit Ihnen sein.

\end{document}
