\documentclass[a4paper,12pt]{article}
\usepackage{a4wide}
\usepackage{tikz}
\usetikzlibrary{calc}
\usepackage{hyperref}

\begin{document}
\pagestyle{empty}
\setlength{\parindent}{0em}
\section*{\noindent Counter }
Your task is to program the behavior of an entity called ``counter". This entity is declared in the attached file ``counter.vhdl" and has the following properties:

\begin{itemize}
	\item Input: CLK with type std\_logic
	\item Input: RST with type std\_logic
	{{enable_property_desc}}
	{{sync_property_desc}}
	{{async_property_desc}}
	{{input_property_desc}}
	{{overflow_property_desc}}
	\item Output: Output with type std\_logic\_vector of length {{counter_width}}
\end{itemize}

\begin{center}
\begin{tikzpicture}
{# draw the entity block dynamically #}


\draw node [draw,rectangle, minimum height={{minimum_height}}mm, minimum width=35mm,rounded corners=2mm,thick](entity){};

    
    \draw[->] ($ (entity.west) + (-10mm, {{ (minimum_height / 2) - current_tikz_offset}} mm)$) -- ($ (entity.west) + (0mm,{{(minimum_height / 2) - current_tikz_offset}}mm)$);
    \draw[anchor=east] node at ($ (entity.west) + (-9mm,{{(minimum_height/2) - current_tikz_offset}}mm)$){ {{input_names[i-1]}} };



    
    \draw[->] ($ (entity.east) + (0mm,{{(minimum_height / 2) - current_tikz_offset}}mm)$) -- ($ (entity.east) + (10mm,{{(minimum_height / 2) - current_tikz_offset}}mm)$);
    \draw[anchor=west] node at ($ (entity.east) + (9mm,{{(minimum_height/2) - current_tikz_offset}}mm)$){ {{output_names[i-1]}} };


\draw node at ($ (entity) - (0,0mm)$){ counter };
\end{tikzpicture}
\end{center}


Do not change the file ``counter.vhdl".\\

The ``counter" entity shall increment the Output vector on {{every_a}} rising edge of the CLK signal. The input RST shall act as synchronous reset, the initial value of Output after reset shall be ``{{init_value_padded}}". When the input Sync{{sync_variation}} is set to `1' then the Output vector shall be set to {{sync_text}} at the rising edge of the CLK signal. When the input Async{{async_variation}} is set to `1' then the Output vector shall be set to {{async_text}} immediately. {{Enable_Overflow_text}} \\

This behavior has to be programmed in the attached file ``counter\_beh.vhdl".\\


To turn in your solution write an email to {{SUBMISSIONEMAIL}} with Subject ``Result Task {{TASKNR}}" and attach your behavior file ``counter\_beh.vhdl".

\vspace{0.7cm}
Good Luck and May the Force be with you.

\end{document}
