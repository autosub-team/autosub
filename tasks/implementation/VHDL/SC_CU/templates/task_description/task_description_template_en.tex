\documentclass[a4paper,12pt]{article}
\usepackage{a4wide}
\usepackage{tikz}
\usetikzlibrary{calc}
\usepackage{hyperref}
\usepackage{color}
\usepackage{pdflscape}
\usepackage{../static/bytefield} % relative to tmp

\newlength{\maxheight}
\setlength{\maxheight}{\heightof{W}}
\newcommand{\baselinealign}[1]{ %
	\centering
	\raisebox{0pt}[\maxheight][0pt]{ #1}%
}

\newcommand\Tstrut{\rule{0pt}{2.6ex}}       % "top" strut

\definecolor{grau}{gray}{.5}

\begin{document}
\pagestyle{empty}
\setlength{\parindent}{0em}
\section*{Single Cycle Control Unit}

Your task is to program the behavior of an entity called ``SC\_CU". This entity is declared in the attached file ``SC\_CU.vhdl" and has the following properties:
\begin{itemize}
\item Input:  Opcode with type std\_logic\_vector of length 6
\item Input:  Funct with type std\_logic\_vector of length 6
\item Input:  Zero with type std\_logic
\item Output: ALUControl with type std\_logic\_vector of length 3
\item Output: Control signals RegDst, Branch, Jump, MemRead, MemtoReg, MemWrite, ALUSrc and RegWrite all with type std\_logic
\end{itemize}

\begin{center}
\begin{tikzpicture}
\draw node [draw,rectangle, minimum height=55mm, minimum width=35mm,rounded corners=2mm,thick](entity){};

\draw[->] ($ (entity.west)+(-10mm,10mm)$) -- ($ (entity.west) + (0mm,10mm)$);
\draw node at ($ (entity.west)-(18mm,-10mm)$){Opcode};

\draw[->] ($ (entity.west)-(10mm,0mm)$) -- ($ (entity.west) - (0mm,0mm)$);
\draw node at ($ (entity.west)-(16mm,0mm)$){Funct};

\draw[->] ($ (entity.west)-(10mm,10mm)$) -- ($ (entity.west) - (0mm,10mm)$);
\draw node at ($ (entity.west)-(15mm,10mm)$){Zero};


\draw[->] ($ (entity.east) + (0mm,24mm)$) -- ($ (entity.east) + (10mm,24mm)$);
\draw node at ($ (entity.east) + (18mm,24mm)$){RegDst};

\draw[->] ($ (entity.east) + (0mm,18mm)$) -- ($ (entity.east) + (10mm,18mm)$);
\draw node at ($ (entity.east) + (18mm,18mm)$){Branch};

\draw[->] ($ (entity.east) + (0mm,12mm)$) -- ($ (entity.east) + (10mm,12mm)$);
\draw node at ($ (entity.east) + (17mm,12mm)$){Jump};

\draw[->] ($ (entity.east) + (0mm,6mm)$) -- ($ (entity.east) + (10mm,6mm)$);
\draw node at ($ (entity.east) + (21mm,6mm)$){MemRead};

\draw[->] ($ (entity.east) + (0mm,0mm)$) -- ($ (entity.east) + (10mm,0mm)$);
\draw node at ($ (entity.east) + (21.5mm,0mm)$){MemtoReg};

\draw[->] ($ (entity.east) + (0mm,-6mm)$) -- ($ (entity.east) + (10mm,-6mm)$);
\draw node at ($ (entity.east) + (21.3mm,-6mm)$){MemWrite};

\draw[->] ($ (entity.east) + (0mm,-12mm)$) -- ($ (entity.east) + (10mm,-12mm)$);
\draw node at ($ (entity.east) + (22.5mm,-12mm)$){ALUControl};

\draw[->] ($ (entity.east) + (0mm,-18mm)$) -- ($ (entity.east) + (10mm,-18mm)$);
\draw node at ($ (entity.east) + (18.5mm,-18mm)$){ALUSrc};

\draw[->] ($ (entity.east) + (0mm,-24mm)$) -- ($ (entity.east) + (10mm,-24mm)$);
\draw node at ($ (entity.east) + (20mm,-24mm)$){RegWrite};

\draw node at ($ (entity) - (0,0mm)$){SC\_CU};

\end{tikzpicture}
\end{center}

Do not change the file ``SC\_CU.vhdl".\\

You will have to implement the following types of instructions:
{{SELECTED_INSTRUCTIONS_TYPE}}

The ``SC\_CU" entity shall control the single cycle processor depicted in  Figure~1 to perform the following instructions:

\begin{table}[h!]
\centering
    \begin{tabular}{|c|c|c|c|c|} \hline \Tstrut
		instruction & opcode  & funct	& zero & type   \\ \hline \Tstrut
		{{SELECTED_INSTRUCTIONS}}
    \hline
    \end{tabular}
\end{table}

{{SELECTED_INSTRUCTIONS_TEXT}}

\begin{table}[h!]
\centering
    \begin{tabular}{|c|c|} \hline \Tstrut
		ALUControl & Function   \\ \hline \Tstrut
		{{SELECTED_ALUCONTROLS}}
    \hline
    \end{tabular}
    \caption{ALUControls}
    \label{tab:ALUControls}
\end{table}

To get a better understanding of the control signals, here is a description of each control signal. Use Figure~1 to see how the control signals control the data paths.
\begin{itemize}

	\item{RegDst: Selects whether the register destination comes from the bits 20 -- 16 or bits 15 -- 11 of the instruction. In other words, this control signal selects if the destination register comes from rt or rd.}

	\item{Branch: Has an effect on the source for the next program counter value. If the branch condition is met it will be `1'. If the current instruction is not a branch instruction it will always be set to `0'.}

	\item{Jump: Has an effect on the source for the next program counter value. If the current instruction is not a jump instruction it will always be set to `0'.}

	\item{MemRead: When the MemRead signal is set to `1', then the data memory outputs the data specified by the read address input.}

	\item{MemtoReg: Selects whether the register's write data input will come from the data memory or the ALU output.}

	\item{MemWrite: When the MemWrite signal is set to `1', then the data memory writes its write data to the write address. }

	\item{ALUControl: Selects the ALU operation the ALU performs on its two input values. The controls for the available operations are listed in Table~1.}

	\item{ALUSrc: Selects whether the ALU gets a value from the register's read data 2 output, or from the sign extended immediate value of the instruction.}

	\item{RegWrite: When the RegWrite signal is set to `1', then the register writes the write data to the write register.\\}

\end{itemize}

Consider which actions each part of the processor in Figure~1 has to take to fulfill the functions of the operations and set the control signals accordingly. This behavior has to be programmed in the attached file ``SC\_CU\_beh.vhdl".\\

To turn in your solution write an email to {{SUBMISSIONEMAIL}} with Subject ``Result Task {{TASKNR}}" and attach your file ``SC\_CU\_beh.vhdl".

\vspace{0.7cm}
Good Luck and May the Force be with you.

\begin{landscape}
\begin{figure}[!h]
\vspace{-0.5cm}
\hspace{-1.8cm}
\includegraphics[width=25.5cm]{../static/Single_Cycle_Processor_V_3_0}% relative to tmp
\caption{Single cycle processor}
\label{fig:SingleCycleProcessor}
\end{figure}
\end{landscape}


\end{document}
